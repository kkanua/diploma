\documentclass[a4paper, 14pt]{extarticle}
\usepackage[utf8]{inputenc}
\usepackage[english,russian]{babel}
\usepackage{indentfirst}

\usepackage[utf8]{inputenc}%включаем свою кодировку: koi8-r или utf8 в UNIX, cp1251 в Windows
\usepackage[english,russian]{babel}%используем русский и английский языки с переносами
\usepackage{amssymb,amsfonts,amsmath,mathtext,enumerate,float} %подключаем нужные пакеты расширений
\usepackage{graphicx}
\usepackage{cmap}
\usepackage{verbatim}
\usepackage{wrapfig}
%\usepackage[T2A]{fontenc}
\graphicspath{{imgs/}}%путь к рисункам
%\usepackage{enumitem}
%\setlist{nolistsep}

\usepackage{geometry} % Меняем поля страницы
\geometry{left=3cm}% левое поле
\geometry{right=2cm}% правое поле
\geometry{top=2cm}% верхнее поле
\geometry{bottom=2cm}% нижнее поле

\geometry{marginparwidth=2.5cm} % Ширина заметок на полях.

\usepackage{cite}

\makeatletter
\renewcommand{\@biblabel}[1]{#1.} % Заменяем библиографию с квадратных скобок на точку:
\makeatother

\usepackage{float}
\usepackage{xcolor}
\usepackage{marginnote}
\reversemarginpar % Перенос заметок с правого поля на левое.

\usepackage{tabularx}

% Полезные типы колонок: колонки с выравниваем влево/по центру/справа 
% фиксированной ширины (как p{}, но не justify).
\newcolumntype{L}[1]{>{\raggedright\let\newline\\\arraybackslash\hspace{0pt}}m{#1}}
\newcolumntype{C}[1]{>{\centering\let\newline\\\arraybackslash\hspace{0pt}}m{#1}}
\newcolumntype{R}[1]{>{\raggedleft\let\newline\\\arraybackslash\hspace{0pt}}m{#1}}

% Колонки для tabularx
\newcolumntype{E}[1]{>{\hsize=#1\hsize\raggedright\arraybackslash}X}%
\newcolumntype{F}[1]{>{\hsize=#1\hsize\centering\arraybackslash}X}%
\newcolumntype{G}[1]{>{\hsize=#1\hsize\raggedleft\arraybackslash}X}%
%\newcolumntype{C}[2]{>{\hsize=#1\hsize\columncolor{#2}\centering\arraybackslash}X}%

\usepackage{indentfirst}
\usepackage{setspace}
\onehalfspacing

% % % % % % % % % % % % % % % % % % %
% Нумерация абзацев
% % % % % % % % % % % % % % % % % % %

\renewcommand{\theenumi}{\arabic{enumi}}% Меняем везде перечисления на цифра.цифра
%\renewcommand{\labelenumi}{\arabic{enumi}}% Меняем везде перечисления на цифра.цифра
\renewcommand{\theenumii}{.\arabic{enumii}}% Меняем везде перечисления на цифра.цифра
\renewcommand{\labelenumii}{\arabic{enumi}.\arabic{enumii}.}% Меняем везде перечисления на цифра.цифра
\renewcommand{\theenumiii}{.\arabic{enumiii}}% Меняем везде перечисления на цифра.цифра
\renewcommand{\labelenumiii}{\arabic{enumi}.\arabic{enumii}.\arabic{enumiii}.}% Меняем везде перечисления на цифра.цифра

\renewcommand{\thesection}{\arabic{section}}
%\renewcommand{\thesubsection}{\arabic{subsection}}
%\renewcommand{\thesubsubsection}{\arabic{subsubsection}}
\setlength{\intextsep}{1.75\intextsep}
\setcounter{secnumdepth}{5}
\setcounter{tocdepth}{5}
%\addtolength{\parskip}{8pt}


% % % % % % % % % % % % % % % % % % %
% Новые команды
% % % % % % % % % % % % % % % % % % %

\newcommand{\off}[1]{{\color{gray}#1}}

%\newcommand{\todo}[1]{\marginpar{\color{red} \footnotesize #1}}
%\newcommand{\todonote}[1]{\todo{#1}}
%\newcommand{\toask}[1]{\marginpar{\color{blue} \footnotesize #1}}

%\newcommand{\todo}[1]{}
%\newcommand{\todonote}[1]{{\color{red}#1}}
%\newcommand{\toask}[1]{{\color{blue}#1}}

\newcommand{\todo}[1]{}
\newcommand{\todonote}[1]{}
\newcommand{\toask}[1]{}


\newcommand{\comm}[1]{\marginpar{\color{red}\Large \textbf{!!}}}
\newcommand{\alt}[2]{$\overset{\text{#2}}{\text{#1}}$}
\newcommand{\tweakedsim}{{\raise.17ex\hbox{$\scriptstyle\sim$}}}
\newcommand{\ceil}[1]{\left\lceil#1\right\rceil}
\newcommand{\abs}[1]{\left|#1\right|}
\renewcommand{\phi}{\varphi}

% % % % % % % % % % % % % % % % % % %
% Библиография
% % % % % % % % % % % % % % % % % % %

%\usepackage[backend=biber,style=unsrt]{biblatex}
%\addbibresource{bibliography.bib}

%\bibliographystyle{unsrt}

% https://tex.stackexchange.com/questions/26575/bibtex-how-can-i-automatically-reduce-long-author-lists-to-xxx-et-al
%\bibliographystyle{unsrtLimauthors.bst}
\bibliographystyle{ugost2008}


\begin{document}
\begin{center}
  \textbf{Аннотация на выпускную квалификационную работу бакалавра Каня Кирилла Олеговича\\ "Разработка программного обеспечения для онлайн монитора светимости детектора Belle II".}
\end{center}\par
    В 2018 г. на ускорительном комплексе SuperKEKB начался эксперимент Belle II проектная светимость которого $8\cdot10^{35} cm^{-2} c^{-1}$. Один из способов контроля набора данных и проверки корректности работы ускорителя является измерение светимости. Для этого в институте ядерной физики был разработан онлайн монитор светимости, который измеряет скорости счета событий $e^-e^+$ рассеяния с электромагнитного калориметра. \par
    Целью данной работы является написание ПО для монитора светимости, которое будет обеспечивать первичную проверку качества, отображение, архивирование и передачу данных. Многопоточное приложение было реализовано на языке программирования Python. Для передачи данных в систему медленного контроля EPICS использовалась библиотека pythonIOC. \par
Для детального анализа работы систем был реализован расчет интегральной светимости за характерные промежутки времени. Для проверки работы калориметра добавлен расчет значений пьедесталов для каждого сектора в режиме реального времени. Для повышения стабильности работы ПО реализовано сохранение текущего состояния в базу данных (БД), при перезапуске данные автоматически восстанавливаются. Также в БД записываются значения светимости с предыдущих заходов для последующего анализа. Для удобства отслеживания работы онлайн монитора светимости была создана программа для визуализации, позволяющая получить все необходимые параметры и данные с монитора светимости. Графический интерфейс был реализован на языке программирования C++ с использование фреймворка Qt. Для получения корректных данных необходимо периодически проводить калибровку монитора светимости. В связи с этим возникла задача автоматизировать данный процесс: расширен протокол управления монитором светимости, что позволило параллельно устанавливать конфигурацию монитора и калориметра и управлять чтением данных с соответствующих модулей. Калибровочные коэффициенты сохраняются в БД и автоматически считываются при запуске монитора светимости. Также реализована автоподстройка калибровочных коэффициентов при изменении аттенюаторных на соответственную величину.\par
В результате данной работы, реализовано ПО для монитора светимости, которое позволяет контролировать процесс работы ускорителя и детектора, а также проверять, сохранять и отображать данные с монитора светимости. 
\begin{flushright}
  \rule{3cm}{.5pt}/Каня К. О.
\end{flushright}
\end{document}
