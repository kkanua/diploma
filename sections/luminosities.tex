  Модуль онлайн монитор светимости измеряет скорость счета $e^+e^-$ рассеяния с торцевых частей калориметра. Однако, требуется получать значения в абсолютных единицах. Таким образом необходимо реализовать перевод из относительных единиц в абсолютные.\par
 Светимость коллайдера через скорость счета $e^+e^-$ рассеяния выражается следующим образом:
\begin{equation}
  L = \frac{1}{\epsilon\sigma}\frac{dN}{dt} 
\end{equation}
где $L$ светимость коллайдера, $\frac{dN}{dt}$ скорость счета, а произведение $\sigma\cdot\epsilon$ видимое сечение регистрации ($\sigma_{vis}$). Следовательно, имея скорость счета и видимое сечение регистрации, можно реализовать расчет светимости в абсолютных единицах. В результате моделирования методом Монте-Карло было установлено, что видимое сечение составляет $\sigma_{vis}=$30.23 нб.\par
  Согласно сценариям ПО, после расчета светимости в абсолютных единицах необходимо реализовать расчет интегральных и максимальных светимостей за следующие промежутки времени: эксперимент, заход, дежурство, Phase 3, день, час и усредненное значение за последние 20 секунд.
 Расчет интегральных светимостей происходит следующим образом: вычитывающее ПО считывает значение скорости счета $e^+e^-$ рассеяния, затем производится перевод текущего значения в абсолютные единицы и добавляется к каждому значению интегральной светимости. Расчет максимальных значений происходит аналогично, только текущее значение сравнивается с предыдущим максимальным значением, если текущее больше, то оно становится новым максимальным значением. Проверка максимальных значений производится для каждого промежутка времени. Промежутки, для которых рассчитываются максимальные значения светимостей, аналогичны промежуткам расчета интегральных светимостей.\par
  Для взаимодействия с системой медленного контроля EPICS используется библиотека pythonIOC, которая позволяет создавать EPICS PV и взаимодействовать с ними.\par
  Одним из требований к ПО является обеспечение сохранности данных при перезагрузках. Таким образом, необходимо дополнительно сохранять значения интегральных и максимальных светимостей и сопутствующую информацию о времени последнего сохранения. В качестве хранилища данных значений была выбрана база данных sqlite в связи с легкой поддержкой данной БД и отсутствием дополнительных подключений к удаленному хосту. В данной БД было создано две таблицы, одна для сохранения всех текущих значений интегральных и максимальных светимостей и сопутствующих данных о временном промежутке когда они были сохранены. Вторая таблица содержит значения светимостей с предыдущих заходов.
