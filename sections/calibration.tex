  Формы сигналов, которые получает монитор светимости имею размерность в единицах ацп на канал. Необходимо получать значения в энергитических единицах. Следовательно, стоит задача определения цены деления ацп для каждого канала. Для этой цели была разработана процедура энергетической калибровки онлайн монитора светимости. Процедура основывается на посылке тестового сигнала и сравнения амплитуд с монитора светимости и уже откалиброванной системой DAQ. Калибровочные коэффициенты расчитываются следующим образом:
\begin{equation}
  k = \frac{A_{DAQ}}{A_{LOM}}
\end{equation}
 где $A_{DAQ}$ и $A_{LOM}$ амплитуды с системы сбора данных DAQ и монитора светимости соответственно.\par
  Подробная схема данной процедуры представлена на рисункеХ и заключается в следующем:
\begin{enumerate}
  \item На электромагнитный калориметр при помощи генератора импульсов, подается тестовый сигнал
  \item Далее сигнал проходит все системы до монитора светимости как реальный сигнал
  \item На выходе с монитора светимости получаем амплитудные значения в каналах АЦП
  \item Далее вычисляются калибровочные коэффициенты по формулеХ
\end{enumerate} 

