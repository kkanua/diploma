  Электромагнитный калориметр является одной из основных частей детектора Belle II. Поскольку треть продуктов распада B-мезонов - это $\pi^0$ или другие нейтральные частицы, которые распадаются на фотоны и регистрируются электромагнитным калориметром. В качестве сцинтилляционного кристалического материала для калориметра Belle II был выбран CsI (Tl) из-за его высокой светоотдачи, относительно короткой длины излучения, хороших механических свойств и умеренной цены. Электромагнитный калориметр состоит из 8736 CsI (Tl) кристаллов и разделен на три части: две торцевые части передняя (Forward) и задняя (Backward), а также цилиндрическая (Barrel). Основными задачами калориметра являются:
\begin{itemize}
  \item Регистрация фотонов и электронов
  \item Идентификация электронов
  \item Работа в широком диапазоне энергий 20 МэВ - 4 ГэВ в лабораторной системе отсчета
  \item Измерение светимости офлайн и онлайн
\end{itemize}\par
  Для регистрации сигнала с кристаллов на каждом из них установлено 2 фотодиода котрые преобразуют сцинтилляционное свечение в электрический сигнал, который поступает на вход предусилителей и преобразуется в импульс напряжения. Далее импульс напряжения поступает в модуль ShaperDSP, который осуществлет формирование, усилиение сигнала, вычисляет амплитуду и время срабатывания счетчика. В каждый модель ShaperDSP поступает сигнал с 8-16 кристаллов, образуя тем самым триггерную ячейку. Каждая торцевая секция разделена на 16 секторов, каждый сектор содержит две триггерные ячейки, данные с которых проходя через ShaperDSP передаются в модуль FAM. FAM суммирует сигналы с двух триггерных ячеек и передает аналоговую сумму сигналов на модуль онлайн монитора светимости (LOM).
