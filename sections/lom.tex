  Одной из целей эксперимента Belle II является достижение проектной светимости равной $8\cdot10^{35}$с$^{-1}$см$^{-2}$. Такая огромная светимость поставила строгие ограничения на детектор, в связи с чем потребовалось произвести множество улучшений. Также в процессе работы при таких значениях светимости необходимо тщательно контроллировать процесс набора данных и иметь обратную связь с ускорителем. Для данной цели используются три онлайн монитора светимости: LumiBelle2, zero degree luminosity monitor (ZDLM) и онлайн монитор светимости (ECL LOM), на улучшение которого и направлена данная работа. Модуль LOM был разроботан в ИЯФ СО РАН и произведен в 2016 году. Ключевой особенностью LOM является то, что данный модуль способен измерять светимость в абсолютных единицах, в отличие от двух других мониторах светимости, которые работают только в относительных единицах.\par 
  Монитор светимости направлен на измерение скорости счета $e^+e^-$ рассеяния с торцевых частей электромагнитного калориметра, произошедшего в коллинеарных секторах торцевых частей. 
