  При проведении эксперимента Belle II важно контроллировать светимость ускорителя, для этой цели в ИЯФ СО РАН был разработан модуль онлайн монитор светимости. Данный модуль измеряет скорость счета $e^+e^-$ рассеяния с торцевых частей электромагнитного калориметра. Измерение светимости в режиме реального времени позволяет получать обратную связь с ускорителем для настройки оптимальных парметров самого ускорителя. Также измерение интеграла светимости позволяет мониторировать процесс набора данных, что является важной задачей при изучении редких распадов, таких как распады $B-$ и $D-$мезонов.
  Данные с монитора светимости, представляют интерес нескольким группам участвующим в эксперименте Belle II. Каждой группе необходимы различные данные и различная степень детализации этих данных. Следовательно, необходимо ПО, которое позволит каждой группе иметь доступ к этим данным. Таким образом, группы и сценарии использования, каждой группой можно описать следующим образом:
\begin{itemize}

  \item Группа операторы ускоритея. Для получения обратной связи с ускорителя, с целью подбора оптимальных значений, данной группе требуются следующие значения:
    \begin{itemize}
      \item Мгновенная ускорительная светимость
      \item Интегральные ускорительные светимости за различные промежутки времени
    \end{itemize}

  \item Группа операторы детектора. Для анализа эффективности работы детектора, данной группе необходимы следующие значения:
    \begin{itemize}
      \item Получать мгновенную детекторную и ускорительную светимости
      \item Получать аналогичные интегральные светимости за различные промежутки времени 
    \end{itemize}

  \item Группа экспертов по ECL? Для отслеживания корректности работы электромагнитного калориметра данной группе необходимо получать следующие данные:
    \begin{itemize}
      \item Значения пьедесталов для каждого сектора
      \item Формы сигналов с каждого сектора 
      \item Амплитудные гистограммы для каждого сектора 
    \end{itemize}

  \item Группа экспертов по обработке данных. Для отбора заходов по светимостям данной группе необходимы следующие значения:
    \begin{itemize}
      \item Интегральные светимости по заходам
    \end{itemize}

  \item Группа руководителей эксперимента. Данной группе для отслеживания достижения проектных значений светимостей необходимо получать следующие данные:
    \begin{itemize}
      \item Значения интегральных светимостей
      \item Значения максимальных светимостей
    \end{itemize}

\end{itemize}
