% Добавить изображение SuperKEKB
  Эксперимент Belle был направлен на изучение распадов B-мезонов и на подтверждение CP-нарушения предсказанного Макото Кобаяши и Тосихидэ Масакава, которые были награждены Нобелевской премией за данное открытие. Считается, что CP-нарушение является одной из причин наблюдаемого доминирования вещества над антиматерией в нашей нынешней вселенной. Однако измеренный уровень CP-нарушения далеко не достаточен для количественного объяснения фактической асимметрии. Следовательно необходимо более детально изучение связных явлений. Новый эксперимент Belle II направлен на поиск новой физики, поиск новых источник CP-нарушения и постановку более строгих ограничений на стандартную модель.\par 
Коллайдер SuperKEKB, расположенный в лаборатории высоких энергий KEK, представляет собой ускоритель с ассиметричной энергией пучков ($E_{e^-}=7$ ГэВ и $E_{e^+}=4$ ГэВ). Данный коллайдер является модернизированной версией B-фабрики KEKB, использовавшейсяя в предыдущем эксперименте Belle. Проектная светимость коллайдера составляет $8\cdot10^{35}$с$^{-1}$см$^{-2}$, что в 40 раз превышает значение достигнутое в предыдущем эксперименте Belle. Такая светимость достигается за счет уменьшения поперечного размера пучка, а также за счет большого угла столкновения пучков. В новом эксперименте Belle планируется набрать в 50 раз больше данных.\par
  Поскольку электронн-позитронные столкновения будут происходить с гораздо большей скоростью, необходимо было модернизировать детектор.
