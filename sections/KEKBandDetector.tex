Коллайдер SuperKEKB, расположенный в лаборатории высоких энергий KEK, представляет собой ускоритель с ассиметричной энергией пучков ($E_{e^-}=7$ ГэВ и $E_{e^+}=4$ ГэВ). Проектная светимость коллайдера составляет $8\cdot10^{35}$с$^{-1}$см$^{-2}$. Такая светимость достигается за счет уменьшения поперечного размера пучка, а также за счет большого угла столкновения пучков. 
