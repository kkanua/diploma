  В эксперименте Belle II используются две системы медленного контроля Network Shared Memory 2 (NSM2) и Experimental Physics and Industrial Control System (EPICS). NSM2 является внутренней разработкой Belle II и основными задачами, возлагаемыми на данную систему, являются\cite{TechRep}:
\begin{itemize}
  \item Контролировать последовательность запуска и остановки всей системы сбора данных
  \item Получать сообщения и статусы от подсистем системы сбора данных DAQ
  \item Контролировать систему высокого напряжения
  \item Сбор мониторинговой информации
\end{itemize}
NSM2 обеспечивает два основных механизма обмена информацией по локальной сети на основе TCP/IP. Первый - это механизм для синхронизации заданного пространства памяти посредством передачи широковещательных пакетов UDP по различным хостам в сегменте сети. Второй называется NSM-сообщением, механизм для отправки данных, при помощи пакетов TCP в пользовательской программе без каких-либо знаний о механизме TCP/IP или системных вызовах, адресах хоста или сетевых конфигурациях. В эксперименте Belle II данная система медленного контроля используется для сбора данных и контроля электроники детекторных систем\cite{NSM}.\par
  Система медленного контроля имеет клиент-серверную архитектуру и в качестве хранилища используется распределенная база данных. Данные, хранящиеся в базе данных, идентифицируются с использованием уникальных идентификаторов, называемых Process Variables (PVs). Эти PV доступны по специальному программному каналу, который называется Channel Access (CA)\cite{EPICS}. Система медленного контроля EPCIS используется в эксперименте Belle II для контроля систем и параметров ускорителя (светимость, токи пучков, уровень фона и т.д.).
