\documentclass[12pt,a4paper]{scrartcl}
\usepackage[utf8]{inputenc}
\usepackage[english,russian]{babel}
\begin{document}
    \begin{center}
        \textbf{Разработка ПО для онлайн монитора светимости детектора Belle II}
        \bigbreak
        К. О. Каня \\
        Институт ядерной физики им. Г. И. Будкера СО РАН г. Новосибирск
    \end{center}

    В 2018 г. на ускорительном комплексе SuperKEKB начался эксперимент Belle II проектная светимость которого $8 * 10^{35} cm^{-2} c^{-1}$. Калориметр одна из основных подсистем детектрока, которая служит для регистрации координат и энергии частиц. Он состоит из цилиндрической и двух торцевых секций: передней (FWD) и задней (BWD). Один из способов контроля работы ускорителя является измерение светимости. В связи с этим в институте ядерной физики был разработан онлайн монитор светимости который измеряет скорости счета событий $e^-e^+$ рассеяния с электромагнитного калориметра. \par
    Целью данной работы является написание ПО для монитора светимости, которое будет обеспечивать первичную проверку качества, отображение, архивирование и передачу данных. Для написания ПО использовалась версия 2.7 языка программирования python для обратной совместимости с уже существующими модулями. Для передачи данных в систему медленного контроля EPICS использовалась библиотека pythonIOC. \par
    Одним из способов контроля набора статистики и отслеживание корректности работы ускорителя является измерение светимости в режиме реального времени. Для более детального анализа работы систем используется расчет интегральной светимости за характерные промежутки времени. Для проверки работы калориметра добвален расчет значений пьедесталов для каждого сектора в режиме реального времени. Для повышения стабильности работы ПО реализовано сохранение текущего состояния в базу данных (БД), при перезапуске данные автоматически восстанавливаются. Также в БД записываются значения светимости с предыдущих заходов для последующего анализа. Существуют программы визуализации различной степени детализации светимости, параметров ускорителя и других? Для удобства отслеживания работы онлайн монитора светимости была создана программа для визуализации, позволяющая получить все необходимые параметры и данные с монитора светимости. Графический интерфейс был реализован на языке программирования C++ с использование фреймворка Qt. Для получения корректных данных необходимо периодически проводить калибровку монитора светимости. В связи с этим возникла задача автоматизировать данный процесс: в ПО добалена возможность поставить на паузу и возобновить чтение данных, считать статус монитора. Также калибровочные коэффициенты сохраняются в БД и автоматически считываются при запуске монитора светимости. \par
    В результате данной работы, реализовано ПО для монитора светимости, которое позволяет контроллировать процесс работы ускорителя и детектора, а также проверять, сохранять и отображать данные с монитора светимости.
\bigbreak
\begin{center}
    Научный руководитель --- М. А. Ремнев
\end{center} 
\end{document}
