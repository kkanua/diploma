\documentclass[a4paper, 12pt]{article}
%\documentclass[a4paper, 14pt]{extarticle}
\usepackage[utf8]{inputenc}
\usepackage[english,russian]{babel}
\usepackage{indentfirst}

\usepackage[utf8]{inputenc}%включаем свою кодировку: koi8-r или utf8 в UNIX, cp1251 в Windows
\usepackage[english,russian]{babel}%используем русский и английский языки с переносами
\usepackage{amssymb,amsfonts,amsmath,mathtext,enumerate,float} %подключаем нужные пакеты расширений
\usepackage{graphicx}
\usepackage{cmap}
\usepackage{verbatim}
%\usepackage[T2A]{fontenc}
\graphicspath{{images/}}%путь к рисункам

\usepackage{geometry} % Меняем поля страницы
\geometry{left=3cm}% левое поле
\geometry{right=2cm}% правое поле
\geometry{top=2cm}% верхнее поле
\geometry{bottom=2cm}% нижнее поле

\geometry{marginparwidth=2.5cm} % Ширина заметок на полях.

\usepackage{cite}

\makeatletter
\renewcommand{\@biblabel}[1]{#1.} % Заменяем библиографию с квадратных скобок на точку:
\makeatother

\usepackage{float}
\usepackage{xcolor}
\usepackage{marginnote}
\reversemarginpar % Перенос заметок с правого поля на левое.

\usepackage{tabularx}

% Полезные типы колонок: колонки с выравниваем влево/по центру/справа 
% фиксированной ширины (как p{}, но не justify).
\newcolumntype{L}[1]{>{\raggedright\let\newline\\\arraybackslash\hspace{0pt}}m{#1}}
\newcolumntype{C}[1]{>{\centering\let\newline\\\arraybackslash\hspace{0pt}}m{#1}}
\newcolumntype{R}[1]{>{\raggedleft\let\newline\\\arraybackslash\hspace{0pt}}m{#1}}

% Колонки для tabularx
\newcolumntype{E}[1]{>{\hsize=#1\hsize\raggedright\arraybackslash}X}%
\newcolumntype{F}[1]{>{\hsize=#1\hsize\centering\arraybackslash}X}%
\newcolumntype{G}[1]{>{\hsize=#1\hsize\raggedleft\arraybackslash}X}%
%\newcolumntype{C}[2]{>{\hsize=#1\hsize\columncolor{#2}\centering\arraybackslash}X}%

\usepackage{indentfirst}
\usepackage{setspace}
\onehalfspacing

% % % % % % % % % % % % % % % % % % %
% Нумерация абзацев
% % % % % % % % % % % % % % % % % % %

\renewcommand{\theenumi}{\arabic{enumi}}% Меняем везде перечисления на цифра.цифра
%\renewcommand{\labelenumi}{\arabic{enumi}}% Меняем везде перечисления на цифра.цифра
\renewcommand{\theenumii}{.\arabic{enumii}}% Меняем везде перечисления на цифра.цифра
\renewcommand{\labelenumii}{\arabic{enumi}.\arabic{enumii}.}% Меняем везде перечисления на цифра.цифра
\renewcommand{\theenumiii}{.\arabic{enumiii}}% Меняем везде перечисления на цифра.цифра
\renewcommand{\labelenumiii}{\arabic{enumi}.\arabic{enumii}.\arabic{enumiii}.}% Меняем везде перечисления на цифра.цифра

\renewcommand{\thesection}{\arabic{section}}
%\renewcommand{\thesubsection}{\arabic{subsection}}
%\renewcommand{\thesubsubsection}{\arabic{subsubsection}}
\setlength{\intextsep}{1.75\intextsep}
\setcounter{secnumdepth}{5}
\setcounter{tocdepth}{5}
%\addtolength{\parskip}{8pt}


% % % % % % % % % % % % % % % % % % %
% Новые команды
% % % % % % % % % % % % % % % % % % %

\newcommand{\off}[1]{{\color{gray}#1}}

%\newcommand{\todo}[1]{\marginpar{\color{red} \footnotesize #1}}
%\newcommand{\todonote}[1]{\todo{#1}}
%\newcommand{\toask}[1]{\marginpar{\color{blue} \footnotesize #1}}

%\newcommand{\todo}[1]{}
%\newcommand{\todonote}[1]{{\color{red}#1}}
%\newcommand{\toask}[1]{{\color{blue}#1}}

\newcommand{\todo}[1]{}
\newcommand{\todonote}[1]{}
\newcommand{\toask}[1]{}


\newcommand{\comm}[1]{\marginpar{\color{red}\Large \textbf{!!}}}
\newcommand{\alt}[2]{$\overset{\text{#2}}{\text{#1}}$}
\newcommand{\tweakedsim}{{\raise.17ex\hbox{$\scriptstyle\sim$}}}
\newcommand{\ceil}[1]{\left\lceil#1\right\rceil}
\newcommand{\abs}[1]{\left|#1\right|}
\renewcommand{\phi}{\varphi}

% % % % % % % % % % % % % % % % % % %
% Библиография
% % % % % % % % % % % % % % % % % % %

%\usepackage[backend=biber,style=unsrt]{biblatex}
%\addbibresource{bibliography.bib}

%\bibliographystyle{unsrt}

% https://tex.stackexchange.com/questions/26575/bibtex-how-can-i-automatically-reduce-long-author-lists-to-xxx-et-al
\bibliographystyle{unsrtLimauthors.bst}
%\bibliographystyle{ugost2008}


\title{Разработка ПО для онлайн монитора светимости детектора Belle II}
\author{Каня Кирилл\\Новосибирский Государствнный Университет}

\begin{document}
%\maketitle
\input{sections/title.tex}
\newpage

\section*{Аннотация}
Здесь будет аннотация
\newpage

\tableofcontents
\newpage

\section*{Введение}
\addcontentsline{toc}{section}{Введение}
  В 2018 году на ускорительном комплексе SuperKEKB начался эксперимент Belle II, который направлен на изучение CP-нарушения в распадах B и D мезонов, $\tau$-физики, а также на поиск Новой физики. Проектная светимость коллайдера составляет $8\cdot10^{35}$с$^{-1}$см$^{-2}$, что в 40 раз превышает светимость достигнутую в предыдущем эксперименте Belle.\par
  SuperKEKB -- электрон-позитронный коллайдер с ассиметричной энергией пучков (7 и 4 ГэВ соответственно), который является улучшенной версией предыдущего коллайдера KEKB\par
  Одной из основных систем детектора является электромагнитный калориметр (ECL). Он предназначен для регистрации фотонов и электронов в широком диапазоне энергий, измерения их энергии и координат. Также данные с электромагнитного калориметра используются для измерения онлайн и офлайн светимости.\par
  При изучении редких распадов необходимо серьезно контролировать процесс набора данных, а также контролировать корректность работы ускорителя и детектора. Одним из способов контроля набора данных и корректности работы ускорителя является измерние светиомсти. Светимость храктеризует количество столкновений частиц в пучке за единицу времении приходящихся на единицу площади. Для более детального контроля измерение светимости производится в режиме реального времени (онлайн). Для данной цели используется модуль онлайн монитор светимости, который был разработан в ИЯФ СО РАН. Онлайн монитор светимости измеряет скорость счета событий $e^-e^+$ рассеяния с торцевых частей электромагнитного калориметра. Данная работа направлена на разработку программного обеспечения для онлайн монитора светимости, которое будет обеспечивать первичную проверку качества, архивирование, отображение и передачу данных.\par
\section{Эксперимент Belle II}
    \subsection{SuperKEKB и детектор Belle2}
    Коллайдер SuperKEKB, расположенный в лаборатории высоких энергий KEK, представляет собой ускоритель с ассиметричной энергией пучков ($E_{e^-}=7$ ГэВ и $E_{e^+}=4$ ГэВ). Проектная светимость коллайдера составляет *?*. Такая светимость достигается за счет уменьшения поперечного размера пучка, а также за счет большого угла столкновения пучков. 

    \subsection{Электромагнитный калориметр}
      Электромагнитный калориметр является одной из основных частей детектора Belle II. Поскольку треть продуктов распада B-мезонов - это $\pi^0$ или другие нейтральные частицы, которые распадаются на фотоны в диапазоне энергий 20МэВ - 4ГэВ, и регистрируются электромагнитным калориметром, то требуется высокая разрешающая способность электромагнитного калориметра. В качестве сцинтилляционного кристаллического материала для калориметра Belle II был выбран CsI (Tl) из-за его высокого световыхода, относительно короткой длины излучения, хороших механических свойств и умеренной цены. Электромагнитный калориметр состоит из 8736 CsI (Tl) кристаллов и разделен на три части: передняя (Forward), задняя (Backward), а также цилиндрическая (Barrel) (Рис. 3).
Основными задачами калориметра являются\cite{TechRep}:
\begin{itemize}
  \item Регистрация фотонов и электронов
  \item Идентификация электронов
  \item Работа в широком диапазоне энергий 20 МэВ - 4 ГэВ в лабораторной системе отсчета
  \item Измерение светимости офлайн и онлайн
\end{itemize}\par
%\begin{wrapfigure}{l}{0.55\textwidth}
%  \begin{center}
%    \includegraphics[width=0.5\textwidth]{lom_connection.pdf}
%  \end{center}
%  \caption{Схема устройства триггерных ячеек в секторах и передача данных монитору светимости через FAM}
%\end{wrapfigure}
  Для регистрации сигнала с кристаллов на каждом из них установлено 2 фотодиода, которые преобразуют сцинтилляционное свечение в электрический сигнал, который поступает на вход предусилителей и преобразуется в импульс напряжения. Далее импульс напряжения поступает в модуль ShaperDSP, который осуществляет формирование, усиление сигнала, вычисляет амплитуду и время срабатывания счетчика. В каждый модуль ShaperDSP поступает сигнал с 8-16 кристаллов, образуя тем самым триггерную ячейку. Каждая торцевая секция разделена на 16 секторов, каждый сектор содержит две триггерные ячейки, данные с которых, проходя через ShaperDSP, передаются в модуль FAM. FAM суммирует сигналы с двух триггерных ячеек и передает аналоговую сумму сигналов на модуль онлайн монитора светимости (LOM) (Рис. 4).
\begin{figure}
\centering
\begin{minipage}[t]{.5\textwidth}
  \centering
  \includegraphics[width=.95\linewidth]{ECL}
  \caption{Срез электромагнитного калориметра (ECL).}
  \label{fig:test1}
\end{minipage}%
\begin{minipage}[t]{.5\textwidth}
  \centering
  \includegraphics[width=.95\linewidth, height=5cm]{lom_connection.pdf}
  \caption{Схема устройства триггерных ячеек в секторах и передача данных монитору светимости через FAM.}
  \label{fig:test2}
\end{minipage}
\end{figure}

    \subsection{Онлайн монитор светимости}
      Одной из целей эксперимента Belle II является достижение проектной светимости равной $8\cdot10^{35}$с$^{-1}$см$^{-2}$. Такая огромная светимость поставила строгие ограничения на детектор, в связи с чем потребовалось произвести множество улучшений. Также в процессе работы при таких значениях светимости необходимо тщательно контролировать процесс набора данных и иметь обратную связь с ускорителем и детектором для оперативного обнаружения и исправления неисправностей и  настройки оптимальных параметров. Для данной цели используются три онлайн монитора светимости: LumiBelle2, zero degree luminosity monitor (ZDLM) и онлайн монитор светимости (ECL LOM), на улучшение которого и направлена данная работа. Модуль LOM (Рис. 5) был разработан в ИЯФ СО РАН и произведен в 2016 году. 
\begin{figure}[htp]
  \centering
  \includegraphics[width=\textwidth]{LOM_picture}
  \caption{Лицевая панель монитора светимости.}
  \label{fig:galaxy}
\end{figure}
Ключевой особенностью LOM является то, что данный модуль способен измерять светимость в абсолютных единицах, в отличие от двух других мониторов светимости, которые работают только в относительных единицах \cite{LumiBelle2}.\par 
  Монитор светимости направлен на измерение скорости счета $e^+e^-$ рассеяния с торцевых частей электромагнитного калориметра, произошедшего в коллинеарных секторах торцевых частей (Рис. 6).
\begin{figure}[htp]
  \centering
  \includegraphics[width=0.5\textwidth]{lom_illustration.png}
  \caption{Схема энерговыделения в секторах при возникновении события $e^+e^-$ рассеяния.}
  \label{fig:galaxy}
\end{figure}

    \subsection{Система сбора данных DAQ}
    Здесь будет про систему сбора данных

    \subsection{Система медленного контроля}
    Здесь будет про систему медленного контроля

\section{Анализ требований к ПО}
    \subsection{Сценарии использования}
      При проведении эксперимента Belle II важно контроллировать светимость ускорителя, для этой цели в ИЯФ СО РАН был разработан модуль онлайн монитор светимости. Данный модуль измеряет скорость счета $e^+e^-$ рассеяния с торцевых частей электромагнитного калориметра. Измерение светимости в режиме реального времени позволяет получать обратную связь с ускорителем для настройки оптимальных парметров самого ускорителя. Также измерение интеграла светимости позволяет мониторировать процесс набора данных, что является важной задачей при изучении редких распадов, таких как распады $B-$ и $D-$мезонов.
  Данные с монитора светимости, представляют интерес нескольким группам участвующим в эксперименте Belle II. Каждой группе необходимы различные данные и различная степень детализации этих данных. Следовательно, необходимо ПО, которое позволит каждой группе иметь доступ к этим данным. Таким образом, группы и сценарии использования, каждой группой можно описать следующим образом:
\begin{itemize}

  \item Группа операторы ускоритея. Для получения обратной связи с ускорителя, с целью подбора оптимальных значений, данной группе требуются следующие значения:
    \begin{itemize}
      \item Мгновенная ускорительная светимость
      \item Интегральные ускорительные светимости за различные промежутки времени
    \end{itemize}

  \item Группа операторы детектора. Для анализа эффективности работы детектора, данной группе необходимы следующие значения:
    \begin{itemize}
      \item Получать мгновенную детекторную и ускорительную светимости
      \item Получать аналогичные интегральные светимости за различные промежутки времени 
    \end{itemize}

  \item Группа экспертов по ECL? Для отслеживания корректности работы электромагнитного калориметра данной группе необходимо получать следующие данные:
    \begin{itemize}
      \item Значения пьедесталов для каждого сектора
      \item Формы сигналов с каждого сектора 
      \item Амплитудные гистограммы для каждого сектора 
    \end{itemize}

  \item Группа экспертов по обработке данных. Для отбора заходов по светимостям данной группе необходимы следующие значения:
    \begin{itemize}
      \item Интегральные светимости по заходам
    \end{itemize}

  \item Группа руководителей эксперимента. Данной группе для отслеживания достижения проектных значений светимостей необходимо получать следующие данные:
    \begin{itemize}
      \item Значения интегральных светимостей
      \item Значения максимальных светимостей
    \end{itemize}

\end{itemize}

    \subsection{Передаваемая информация}
    \input{sections/data.tex}
    \subsection{Архитектура ПО}
      ПО для онлайн монитора светимости работает на выделенном компьютере. Считывающие ПО запущенное на данном компьютере непрерывно считывает данные из модуля LOM и позволяет получать значения светимости в локальной сети, а также отправляет полученные значения в системы медленного контроля. Поскольку поток данных с монитора светимости составляет приблизительно 35 Кб/с, то не накладывается строгих ограничений на производительность считывающего ПО и ПО мониторинга. В связи с этим Python был выбран в качестве основного языка для написания большей части програмного обеспечения.\par
  Считывающее ПО представляет собой многопоточный TCP-сервер и работает как прокси для модуля монитора светимости. Поскольку программа монитора светимости является однопоточной, одной из важных функций считывающего ПО является обеспечение обработка поступающих параллельно входящих запросов и предотвращение возможных сбоев прошивки, вызванных несколькими одновременными запросами к монитору светимости. Он также кэширует данные, которые могут запрашиваться клиентскими приложениями, тем самым сокращая количество операций считывания и повышая общую производительность.\par
  Для дальнейшей передачи данных сервер LOM использует библиотеку pythonIOC, встроенный сервер доступа к каналам (Channel Access, CA) EPICS, что делает всю информацию о светимости доступной в системе медленного контроля Belle II. Помимо значений мгновенной светимости, он также производит расчет интегральных и максимальных светимостей, значений пьедесталов, обеспечивает сохранность данных о которых будет описано подробнее в следующей главе.
\begin{figure}[htp]
  \centering
  \includegraphics[width=\textwidth]{LOM_software.pdf}
  \caption{Схема архитектуры ПО}
  \label{fig:galaxy}
\end{figure}

    \subsection{Требования к ПО}
      Для корректного выполнения поставленых к ПО задач, необходимо соблюдать требования выдвигаемые к ПО. Так для ПО онлайн монитора светимости, существуют следующие ребования к его функционированию:
\begin{itemize}
  \item Так как данные с монитора светимости приходят с частотой 1 Гц, то необходимо реализовать обработку данных за время меньшее 1 секунды
  \item Необходимо восстановление накопленных данных при перезагрузке ПО  
  \item Необходимо обнаружение и обработка событий при получении некорректных данных 
  \item Необходимо реализовать поддержку систем медленного контроля NSM2 и EPCIS для хранения данных
  \item ПО должно уметь передавать данные по сети TCP/IP, следовательно необходим протокол взаимодействия с LOM
\end{itemize}

\section{Реализация ПО для онлайн монитора светимости}
    \subsection{Архитектура ПО}
      ПО для онлайн монитора светимости работает на выделенном компьютере. Считывающие ПО запущенное на данном компьютере непрерывно считывает данные из модуля LOM и позволяет получать значения светимости в локальной сети, а также отправляет полученные значения в системы медленного контроля. Поскольку поток данных с монитора светимости составляет приблизительно 35 Кб/с, то не накладывается строгих ограничений на производительность считывающего ПО и ПО мониторинга. В связи с этим Python был выбран в качестве основного языка для написания большей части програмного обеспечения.\par
  Считывающее ПО представляет собой многопоточный TCP-сервер и работает как прокси для модуля монитора светимости. Поскольку программа монитора светимости является однопоточной, одной из важных функций считывающего ПО является обеспечение обработка поступающих параллельно входящих запросов и предотвращение возможных сбоев прошивки, вызванных несколькими одновременными запросами к монитору светимости. Он также кэширует данные, которые могут запрашиваться клиентскими приложениями, тем самым сокращая количество операций считывания и повышая общую производительность.\par
  Для дальнейшей передачи данных сервер LOM использует библиотеку pythonIOC, встроенный сервер доступа к каналам (Channel Access, CA) EPICS, что делает всю информацию о светимости доступной в системе медленного контроля Belle II. Помимо значений мгновенной светимости, он также производит расчет интегральных и максимальных светимостей, значений пьедесталов, обеспечивает сохранность данных о которых будет описано подробнее в следующей главе.
\begin{figure}[htp]
  \centering
  \includegraphics[width=\textwidth]{LOM_software.pdf}
  \caption{Схема архитектуры ПО}
  \label{fig:galaxy}
\end{figure}

    \subsection{Интегральные и максимальные значения светимостей}
      Модуль онлайн монитор светимости измеряет скорость счета $e^+e^-$ рассеяния с торцевых частей калориметра. Однако, требуется получать значения в абсолютных единицах. Таким образом необходимо реализовать перевод из относительных единиц в абсолютные.\par
 Светимость коллайдера через скорость счета $e^+e^-$ рассеяния выражается следующим образом:
\begin{equation}
  L = \frac{1}{\epsilon\sigma}\frac{dN}{dt} 
\end{equation}
где $L$ светимость коллайдера, $\frac{dN}{dt}$ скорость счета, а произведение $\sigma\cdot\epsilon$ видимое сечение регистрации ($\sigma_{vis}$). Следовательно, имея скорость счета и видимое сечение регистрации, можно реализовать расчет светимости в абсолютных единицах. В результате моделирования методом Монте-Карло было установлено, что видимое сечение составляет $\sigma_{vis}=$30.23 нб.\par
  Согласно сценариям ПО, после расчета светимости в абсолютных единицах необходимо реализовать расчет интегральных и максимальных светимостей за следующие промежутки времени: эксперимент, заход, дежурство, Phase 3, день, час и усредненное значение за последние 20 секунд.
 Расчет интегральных светимостей происходит следующим образом: вычитывающее ПО считывает значение скорости счета $e^+e^-$ рассеяния, затем производится перевод текущего значения в абсолютные единицы и добавляется к каждому значению интегральной светимости. Расчет максимальных значений происходит аналогично, только текущее значение сравнивается с предыдущим максимальным значением, если текущее больше, то оно становится новым максимальным значением. Проверка максимальных значений производится для каждого промежутка времени. Промежутки, для которых рассчитываются максимальные значения светимостей, аналогичны промежуткам расчета интегральных светимостей.\par
  Для взаимодействия с системой медленного контроля EPICS используется библиотека pythonIOC, которая позволяет создавать EPICS PV и взаимодействовать с ними.\par
  Одним из требований к ПО является обеспечение сохранности данных при перезагрузках. Таким образом, необходимо дополнительно сохранять значения интегральных и максимальных светимостей и сопутствующую информацию о времени последнего сохранения. В качестве хранилища данных значений была выбрана база данных sqlite в связи с легкой поддержкой данной БД и отсутствием дополнительных подключений к удаленному хосту. В данной БД было создано две таблицы, одна для сохранения всех текущих значений интегральных и максимальных светимостей и сопутствующих данных о временном промежутке когда они были сохранены. Вторая таблица содержит значения светимостей с предыдущих заходов.

    \subsection{Расчет пьедесталов}
      Так как монитор светимости работает на базе электромагнитного калориметра, то полученные данные можно использовать для получения обратной связи с электромагнитным калориметром в режиме реального времени. Возможность получения обратной связи полезна тем, что можно оперативно обнаружить неисправности в работе электромагнитного калориметра и устранить их. В качестве обратной связи удобно использовать значения пьедесталов для каждого сектора. В общем случае пьедестал это отклонение уровня сигнала от нулевого положения?\par
  Для каждого из 32 секторов монитор светимости получает аналоговую сумму сигналов с 2 триггерных ячеек. Соответственно, имея форму сигнала, можно определять отклонение от нулевого положения для каждого сектора в режиме реально времени. Форма сигнала представляет собой массив из 2048 значений на один сектор. Расчет происходит следующим образом:
\begin{enumerate}
  \item Считываем формы сигналов с монитора светимости
  \item Находим максимальное значение в массиве
  \item Проверяем превышает ли максимальное значение установленное пороговое (устанавливаем было ли событие в данном секторе)
  \item Если максимальное значение превышает пороговое, то необходимо удалить 50 значений слева от максимума и 150 справа (Рис. 9). Иначе переходим к следующему пункту
  \item Считаем среднее арифметическое по оставшимся значениям
\end{enumerate}
Данный алгоритм повторяем для каждого сектора. Все полученые значения пьедесталов сохраняются в систему медленного контроля EPICS.
\begin{figure}[htp]
  \centering
  \includegraphics[width=\textwidth]{Pedestal}
  \caption{Схема расчета значений пьедесталов}
  \label{fig:galaxy}
\end{figure}

    \subsection{Графический интерфейс}
    Здесь будет про графический интерфейс

    \subsection{Калибровка онлайн монитора светимости}
      Формы сигналов, которые получает монитор светимости имеют размерность в единицах АЦП на канал. Необходимо получать значения в энергетических единицах. Следовательно, стоит задача определения цены деления АЦП для каждого канала. Для этой цели была разработана процедура энергетической калибровки онлайн монитора светимости. Процедура основывается на посылке тестового сигнала и сравнения амплитуд с монитора светимости и уже откалиброванной системой DAQ. Калибровочные коэффициенты рассчитываются следующим образом:
\begin{equation}
  \eta = \frac{A_{DAQ}}{A_{LOM}}
\end{equation}
 где $A_{DAQ}$ и $A_{LOM}$ амплитуды с системы сбора данных DAQ и монитора светимости соответственно.\par
  Подробная схема данной процедуры представлена на рисунке 11 и заключается в следующем:
\begin{enumerate}
  \item На электромагнитный калориметр при помощи генератора импульсов подается тестовый сигнал
  \item Далее сигнал проходит все системы до монитора светимости как реальный сигнал
  \item На выходе с монитора светимости получаем амплитудные значения в каналах АЦП
  \item Далее вычисляются калибровочные коэффициенты по формуле (2)
\end{enumerate}\par
\begin{figure}[htp]
  \centering
  \includegraphics[width=\textwidth]{calibration.png}
  \caption{Процесс энергетической калибровки монитора светимости.}
  \label{fig:galaxy}
\end{figure}
  Так как в процессе энергетической калибровки необходимо параллельно читать данные с монитора светимости и амплитудные значения с системы сбора данных DAQ, то нужно иметь возможность управлять чтением данных с монитора светимости. Требуется останавливать и продолжать чтение данных в любой момент времени. Для реализации данной возможности был расширен протокол монитора светимости. Были добавлены команды, которые позволяют остановить чтение данных с монитора светимости, возобновить чтение данных и посмотреть текущий статус монитора светимости.\par
  Также для автоматизации контроля актуальной версии калибровочных коэффициентов, было реализовано хранение калибровочных коэффициентов в базе данных DAQ. При запуске ПО находит последнюю версию калибровочных коэффициентов и использует их в дальнейшем.\par
  В торцах электромагнитного калориметра находится 2112 кристаллов, однако, монитор светимости в конечном итоге получает суммированную информацию с этих каналов, сгруппированную по 32 секторам. Таким образом форма сигнала для каждого сектора -- это сумма форм сигналов от, в среднем, 2112/32 = 66 каналов. Однако, все кристаллы обладают немного разными физическими характеристиками, в частности, разным световыходом, требуется суммировать сигналы с неким весом. Этот вес называется коэффициентом аттенюации. Проанализировав изменения аттенюаторных и калибровочных коэффициентов, было замечено, что они коррелируют друг с другом. Поэтому, для более детального анализа, было проведен анализ изменения аттенюаторных коэффициентов и калибровочных за одинаковые промежутки времени. Результат сравнения представлен на рисунке 12. Таким образом, можно проверять, если изменились аттенюаторные коэффициенты, то на соответствующую величину изменять и калибровочные коэффициенты. Видно, что для некоторых секторов, к примеру секторов 1 и 28, корреляция не наблюдается, поэтому процедура автоподстройки коэффициентов является только вспомогательной операцией по отношению к калибровке. Тем не менее, эта процедура позволяет точнее исследовать вклад различных факторов, влияющих на цену деления АЦП.\par
  Также в случае возникновения проблем с соединением или считыванием значений из базы данных в компьютере, на котором запущено ПО, имеется офлайн версия калибровочных коэффициентов. Такой подход позволяет функционировать ПО независимо от соединения с БД.
\begin{figure}[htp]
  \centering
  \includegraphics[width=\textwidth]{coefs_new.pdf}
  \caption{Изменение калибровочных и аттенюаторных коэффициентов за равный промежуток времени.}
  \label{fig:galaxy}
\end{figure}

\section*{Заключение}
\addcontentsline{toc}{section}{Заключение}
    В рамках данной работы было улучшено ПО для онлайн монитора светимости:
    \begin{itemize}
        \item Изменена архитектура ПО, что позволило увеличить стабильность работы системы. При помощи библиотеки pythonIOC реализована параллельная передача данных в системы медленного контроля NSM2 и EPICS.
        \item Добавлен расчет интегральной и максимальной светимостей за характерные промежутки времени.
        \item Добавлен расчет значений пьедесталов для каждого сектора, значения высчитываются в режиме реального времени.
        \item Создана база данных на основе sqlite для сохранения текущих значений светимостей, также записываются значение светимостей за предыдущие заходы.
        \item Расширен протокол управления монитором светимости. Реализованы команды pause и continue.
        \item Добавлено считывание значений калибровочных коэффициентов из базы данных при запуске
    \end{itemize}
    
    Также был улучшен графический интерфейс для монитора светимости, который позволяет проводить удаленную настройку параметров, а также визуализирует данные с монитора светимости
    \begin{itemize} 
        \item Добавлено отображение значений пьедесталов для каждого сектора.
        \item Добавлено считывание порогового значения амплитуд для каждого сектора.
        \item Также были исправлены незначительные ошибки и улучшен интерфейс.
    \end{itemize}
    
    Также была написана программа для отображения основных параметров с монитора светимости, которую планируется интегрировать с веб-сервером?
\section*{Список литературы}
\addcontentsline{toc}{section}{Список литературы}

\end{document}
